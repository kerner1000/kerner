\documentclass[%
	pdftex,%              PDFTex verwenden da wir ausschliesslich ein PDF erzeugen.
	a4paper,%             Wir verwenden A4 Papier.
	oneside,%             Einseitiger Druck.
	12pt,%                Grosse Schrift, besser geeignet für A4.
	final, %              draft oder final
	halfparskip,%         Halbe Zeile Abstand zwischen Absätzen.
	%chapterprefix,%       Kapitel mit 'Kapitel' anschreiben.
	headsepline,%         Linie nach Kopfzeile.
	footsepline,%         Linie vor Fusszeile.
	bibtotocnumbered,%    Literaturverzeichnis im Inhaltsverzeichnis nummeriert einügen.
	idxtotoc%             Index ins Inhaltsverzeichnis einfügen.
]{scrbook}

%
%  2. Festlegen der Zeichencodierung des Dokuments und des Zeichensatzes.
%     Wir verwenden 'Latin1' (ISO-8859-1) für das Dokument,
%     und die 'T1' codierung für die Schrift.
%
\usepackage[utf8]{inputenc}
\usepackage[T1]{fontenc}
%
%  3. Packet für die Index-Erstellung laden.
%
\usepackage{makeidx}

%
%  4. Paket für die Lokalisierung ins Deutsche laden.
%     Wir verwenden neue deutsche Rechtschreibung mit 'ngerman'.
%
\usepackage[ngerman]{babel}


%
%  5. Paket für Anführungszeichen laden.
%     Wir setzen den Stil auf 'swiss', und verwenden so die Schweizer Anführungszeichen.
%    Ändern auf französiche?
%
%\usepackage[german=swiss]{csquotes}
\usepackage[babel,german=guillemets]{csquotes}

%
%  6. Paket für erweiterte Tabelleneigenschaften.
%
\usepackage{array}
%


%
%  7. Paket um Grafiken im Dokument einbetten zu können.
%     Im PDF sind GIF, PNG, und PDF Grafiken möglich.

%\usepackage{graphicx}
\usepackage[dvips]{graphicx}
\usepackage{floatflt} 
%
%  8. Pakete für mathematischen Textsatz.
%
\usepackage{amsmath}
\usepackage{amssymb}
%\usepackage{dsfont}
%\usepackage{mathtools}

%
%  9. Paket um Textteile drehen zu können.
%
\usepackage{rotating}

%
% 11. Paket für spezielle PDF features.
%
\usepackage[%
	pdftitle={Diplomarbeit},%                        Titel des PDF Dokuments.
	pdfauthor={Alexander Kerner},%              Autor des PDF Dokuments.
	pdfsubject={subject},%                    Thema des PDF Dokuments.
	pdfkeywords={Diplomarbeit, Alexander Kerner,%           Schlüsselwörter für das PDF.
		 %     (Diese werden von Suchmaschinen
		},%                        auch für PDF Dokumente indexiert.)
	pdfpagemode=UseOutlines,%                                  Inhaltsverzeichnis anzeigen beim Öffnen
	pdfdisplaydoctitle=true,%                                  Dokumenttitel statt Dateiname anzeigen.
	pdflang=de%                                               Sprache des Dokuments.
]{hyperref}

%
%  1. Index erzeugen.
%
\makeindex

%
% E. SILBENTRENNUNG
% Hier kann angegeben werden, wie ein Wort zu trennen ist (z.B.: {Mau-mau Han-dy} )

\hyphenation{}