\chapter{Umsetzung}
\section{Design}
\section{Module}
\subsection{Conrad}
Conrad \index{Conrad} ist
ein \textit{de novo} \index{de novo} gene caller \index{gene caller},
der Genstrukturen wie Exons und Introns vorhersagen kann.
Die Vorhersage stützt sich dabei auf \textit{semi-Markow Conditional Random
Fields} \index{Markow} \index{Conditional Random Fields}, kurz \textbf{CRF}.
\index{CRF|see{Conditional Random Fields}}
\\ Conrad benötigt für seine Genvorhersage eine FASTA-Datei mit einer oder
mehreren zu analysierenden DNA-Sequenzen, sowie eine Binärdatei, die die
Parameter für den CRF-Alorithmus entält.
\\ Diese Binärdatei wird in einem
Trainingslauf erstellt, der vor der eigentlichen Genvorhersage durchgeführt
wird. Hierbei wird der CRF-Algorithmus auf die Eingabesequenz
\enquote{trainiert}, um die Qualität der Genvorhersage zu maximieren.
\\
\htmladdnormallink{http://www.broadinstitute.org/annotation/conrad/}{http://www.broadinstitute.org/annotation/conrad/}
