\chapter{Zielsetzung}
Das Max-Planck Institut für Züchtungsforschung in Köln hatte bisher keine
eigene Pipeline zur Verfügung, um anfallende Genom-Annotationen eigenständig
durchführen zu können. Problemstellungen dieser Art wurden bis dahin von
Personen außerhalb des Institutes durchgeführt.
Ziel dieser Arbeit war es daher, eine Annotationspipeline zu entwerfen, die
innerhalb des Institutes verfügbar ist. Neben der unmittelbaren Verfügbarkeit
\section{LSF}