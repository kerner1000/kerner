% 2009-08-05

\chapter{Zielsetzung}
\todo{pipeline kommt erst später}
Das Max-Planck Institut für Züchtungsforschung in Köln hatte bisher keine
eigene Pipeline zur Verfügung, um anfallende Genom-Annotationen eigenständig
durchführen zu können. Problemstellungen dieser Art wurden bis dahin von
Personen außerhalb des Institutes durchgeführt.
\index{Pipeline|see{Annotationspipeline}}
\index{Annotation}

Ziel dieser Arbeit war es daher, eine Annotationspipeline zu entwerfen, die
innerhalb des Institutes verfügbar ist. Neben der unmittelbaren Verfügbarkeit
der Pipeline standen hier verschiedene technische Anforderungen im Vordergrund:
\index{Annotationspipeline}

Die Pipeline soll generell modular und möglichst generisch strukturiert sein.
Auf diese Weise soll zum einen ermöglicht werden, die Software mit vertretbarem
Aufwand für verschiedene Anforderungen und Einsatzumgebungen zu konfigurieren,
zum Anderen soll die Software möglichst einfach mit zusätzlichen Funktionen
erweiterbar sein.
Zu diesem Zweck soll die Pipeline über ein \first{Plug-in} System verfügen, in
dem Funktionalität über Plug-ins hinzugefügt und konfiguriert werden kann.
\index{Plug-in}
Durch eine entsprechende  Modularisierung wird der Annotationsprozess
ausschließlich über die Menge der genutzten Plug-ins definiert und
konfiguriert.
Der versierte Nutzer kann das System durch das Erstellen eigener Plug-ins um
weitere Verarbeitungsschritte oder andere Funktionen erweitern.

Das Institut verfügt über einen eigenen compute cluster, auf dem sehr zeit- und
ressourcenintensive Prozesse ausgeführt werden können.
\footnote{Load Sharing Facility, \textbf{LSF}, ist ein propritärer job scheduler
der Firma Platform Computing. Er wird genutzt, um batch jobs in einem Unix oder
Windows basiertem Netzwerk auf verschiedene Arbeitsstationen zu
verteilen.\citep{ault_oracle_2004}\htmladdnormallink{Platform LSF}
{http://www.platform.com/Products/platform-lsf}}
\index{LSF|see{Platform LSF}}
\index{Platform LSF}
\index{compute cluster|see{Platform LSF}}
Die Pipeline soll in der Lage sein, einzelne oder mehrere Prozesse auf dem
compute cluster auszuführen, um lokalen Ressourcen möglichst wenig zu
belasten. Der Gebrauch des clusters soll hierbei aber optional sein, um die
Portierbarkeit der Pipeline nicht unnötig einzuschränken.
\index{Pipeline}

