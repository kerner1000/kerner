%% 2009-08-20

\chapter{Zielsetzung}\label{chp:zielsetzung}
Das Max-Planck Institut für Züchtungsforschung in Köln hatte bisher keine
eigene Software zur Verfügung, um anfallende Genomannotationen eigenständig
durchführen zu können. Problemstellungen dieser Art wurden bis dahin von
Personen außerhalb des Institutes durchgeführt.
\index{Max Planck Institut für Züchtungsforschung}
\index{Annotation!Genom}

Ziel dieser Arbeit war es daher, eine Anwendung zu entwerfen, die
innerhalb des Institutes verfügbar ist. Neben der unmittelbaren Verfügbarkeit
der Anwendung standen hier verschiedene andere Anforderungen im Vordergrund:

Der \enquote{Input}, also die Eingangsdaten, die die Software für ihre
Berechungen benötigt, soll möglichst gering gehalten werden, da zu denen am
Institut untersuchten Pflanzen und Pilzen bis dato oft gar keine bis sehr wenige
Daten verfügbar sind.
Der Input soll sich also im Wesentlichen auf die genomische Sequenz selbst, mit
unter auch auf einige EST-Sequenzen
\footnote{\todo{EST}}
beschränken.
\index{Sequenz!genomisch}
\index{EST|see{Expressed Sequence Tag}}
\index{Expressed Sequence Tag}

Die Anwendung soll generell modular und möglichst generisch strukturiert sein.
Auf diese Weise soll zum einen ermöglicht werden, sie mit vertretbarem
Aufwand für verschiedene Anforderungen und Einsatzumgebungen zu konfigurieren,
zum Anderen soll die Anwendung möglichst einfach mit zusätzlichen Funktionen
erweiterbar sein.
Zu diesem Zweck soll die Anwendung über ein \first{Plug-in} System verfügen, in
dem Funktionalität über separate Softwarepackete hinzugefügt und konfiguriert
werden kann.
\index{Plug-in}
Durch eine entsprechende  Modularisierung wird der Annotationsprozess
ausschließlich über die Menge der genutzten Plug-ins definiert und
konfiguriert.
\index{Modularisierung}
Der versierte Nutzer kann das System durch das Erstellen eigener Plug-ins um
weitere Verarbeitungsschritte oder andere Funktionen erweitern.

Das Institut verfügt über einen eigenen compute cluster, auf dem sehr zeit- und
ressourcenintensive Prozesse ausgeführt werden können.
\footnote{Load Sharing Facility, \textbf{LSF}, ist ein propritärer job scheduler
der Firma Platform Computing.
Er wird genutzt, um batch jobs in einem Unix oder
Windows basiertem Netzwerk auf verschiedene Arbeitsstationen zu
verteilen.
\todo{meeeehr}
\todo{weniger denglisch}
\citep{ault_oracle_2004}\htmladdnormallink{Platform
LSF} {http://www.platform.com/Products/platform-lsf}}
\index{LSF|see{Platform LSF}}
\index{Platform LSF}
\index{compute cluster|see{Platform LSF}}
Die Anwendung soll in der Lage sein, einzelne oder mehrere Prozesse auf dem
compute cluster auszuführen, um lokalen Ressourcen möglichst wenig zu
belasten. Der Gebrauch des clusters soll hierbei aber optional sein, um die
Portierbarkeit der Anwendung nicht unnötig einzuschränken. 

Als \name{Proof of Concept} wird die Anwendung in ein aktuelles Projekt
eingegleidert, in dem Große Teile des Genoms von \name{Colletotrichum
higginsianum} annotiert werden sollen. \todo{mehr dazu?}
\index{\name{Colletotrichum higginsianum}}

\todo{Anwendung, Anwendung, Anwendung}
