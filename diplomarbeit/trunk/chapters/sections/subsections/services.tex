\subsection{Services} % 93
Das OSGi Framework stellt gemäß der serviceorientierten Architektur eine
zentrale, bundleübergreifende \first{Service Registry} bereit, an der
sogenannte \first{Services} angemeldet und abgefragt werden können.
\index{Service Registry}
Ein Service ist in diesem Fall ein simples Java-Objekt, typischerweise ein
Interface, das unter dem Interface-Namen und einer optionalen Beschreibung an
der Registry angemeldet wird.
Der Zugriff auf die Registry kann durch jedes beliebige, im
Framework geladene Bundle erfolgen. Der Zugriff kann hier aktiv oder passiv
erfolgen.
\index{Bundle}

Bei einem aktiven Zugriff, also der Registrierung eines Services durch ein
Bundle, wird technisch gesehen ein Service-Namen, sowie eine Beschreibung des
Services an der Service Registry hinterlegt. Der Service-Name ist üblicherweise
der voll qualifizierende Klassenname des Service Interfaces, die Beschreibung
des Service ist eine Sammlung von Strings, die eine entsprechende Beschreibung
der bereitgestellten Funktion beinhalten.
\index{Service Interface}

Bei einem passiven Zugriff durch ein beliebiges Bundle, also dem Abfragen eines
Service über dessen Namen oder Beschreibung, liefert die Registry eine
Referenz auf das entsprechende Interface zurück.
Über diese Referenz auf das Service Interface kann nun die Funktionalität, die
dieses Interface bereitstellt, genutzt werden, ohne dass hierbei bekannt ist,
wie diese Funktionalität implementiert ist, oder welches Bundle diese
bereitstellt.

Durch die dynamische Modularisierung des OSGi Frameworks können Services zur
Laufzeit \enquote{kommen und gehen}. Es liegt in der Verantwortung des
nutzenden Bundles, auf die aktuelle Verfügbarkeit des Service entsprechend zu
reagieren.

Die \first{OSGi Service Platform} enthält aufbauend auf das OSGi Framework eine
Reihe von \first{Standard Services}, mit denen häufig wiederkehrenden
Problemstellungen begegnet werden kann. So steht beispielsweise ein Log-Service
zur Verfügung, über den Bundles Nachrichten absenden und empfangen können.
Desweiteren sind über die Standard Services auch Funktionenen zur
Benutzerverwaltung, HTTP oder Datenbankanbindung realisiert.
\index{OSGi Service Platform}
% Standard Services 13,25
\citep{wtherich_die_2008}