\subsubsection{Bundles}
Ein Bundle ist nach OSGi Spezifikation\citep{osgi_2009} eine technische Einheit
von Klassen und Ressourcen, die eigentständig in der Anwenung gestartet,
gestoppt, installiert und deinstalliert werden kann.
\index{Bundle}

Ressourcen und Klassen eines Bundles können anderen Bundles bereitgestellt
werden. Dazu müssen sie vom bereitstellenden Bundle explizit exportiert werden
und durch das nutzende Bundle explizit importiert werden. Jedes Bundle stellt einen
Bundlenamen, der üblicherweise von der enthaltenen Paketstruktur
abgeleitet wird, sowie eine Versionsnummer bereit. Aus diesen beiden Komponenten
wird dann eine eindeutige Identifizierung des Bundles generiert.
So ist es beispielsweise möglich, ein und das selbe Bundle in unterschiedlichen
Versionierungen zu installieren.

Über die \first{MANIFEST.MF}, die ohnehin in jedem Jar-File vorhanden ist, wird
das Bundle neben dem Namen und der Version mit weiteren Meta-Informationen, wie
importierte und exportierte Pakete oder Lauzeitumgebung, ausgestattet.
\index{MANIFEST.MF}

Jedes Bundle ist innerhalb des Frameworks mit einem eigenen Class-Loader
ausgestattet, über welchen ausschliesslich die Bundle-eigenen Klassen geladen
werden. Auf diese Weise sind die einzelen Bundles strikt voneinander getrennt und die
Import-Export-Beziehungen zwischen den Bundles können explizit gesteuert
werden. 
Neben anderen Vorteilen, wie etwa das Installieren des selben Bundles in
unterschiedlichen Versionen, wird auf diese Weise das vollständig
dynamische Hinzufügen und Entfernen von Bundles erst ermöglicht, da der
Class Path des Class Loaders nach dessen Instanziierung nicht mehr verändert
werden kann.
\citep{wtherich_die_2008}
% selbe version 17
% importieren, exportieren 13,79
% class loading 89
% MANIFEST.MF 21
% lebenszyklen 23,55
% unterschied Bundle / Plug-In 41