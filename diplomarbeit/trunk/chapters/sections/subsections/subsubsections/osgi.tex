\subsubsection{Das OSGi Framework}\label{chp:osgi}

Die \name{OSGi Alliance}, früher \enquote{\textit{Open Services Gateway
initiative}}, ist ein Zusammenschluß verschiedener Unternehmen, wie z.B. IBM,
Oracle oder Sun Microsystems.
Sie spezifiziert die \first{OSGi Service Platform}, eine Java-basierte
Softwareplattform, die nach einem \first{Komponentenmodell}
% footnote is von wikipedia !!!
\footnote{nach Gruhn und Thiel\citep{gruhn_komponentenmodelle_2000}:
\enquote{Ein Komponentenmodell legt einen Rahmen für die Entwicklung [..] von
Komponenten fest, der strukturelle Anforderungen hinsichtlich Verknüpfungs-
bzw. Kompositionsmöglichkeiten sowie verhaltensorientierte Anforderungen
hinsichtlich Kollaborationsmöglichkeiten an die Komponenten stellt.}}
organisiert ist.
\citep{wtherich_die_2008}
Einzelne Komponenten, sogenannte \first{Bundles}, können der Anwendung
dynamisch hinzugefügt und wieder entfernt werden ohne dass ein erneutes
Kompilieren oder Starten der Anwendung nötig ist.
Abhängigkeiten zwischen Bundles werden dabei automatisch aufgelöst; ein
intelligentes Versionsmanagement steht ebenfalls zur Verfügung.

Die \name{OSGi Service Platform} weisst ausserdem eine serviceorientierte
Architektur \footnote{Unter \first{serviceorientierte Architektur} oder auch
\first{dienstorientierte Architektur} versteht man ein Softwaredesign, dass
sich durch sogenannte Dienste oder auch Services auszeichnet. Diese können
appliktaions-global an einer Registry angemeldet und abgefragt werden. Sie
stellen dabei eine Schnittstelle zu bestimmten Funktionen bereit, ohne dabei
die zu Grunde liegende Implementierung preiszugeben.}
auf: 
Sie stellt eine globale Registry (\first{Service Registry})
bereit, an der Bundles \first{Dienste} bzw. \first{Services} anmelden und
abfragen können.

\subsection{Bundles}
Ein Bundle ist nach OSGi Spezifikation\citep{osgi_2009} eine technische Einheit
von Klassen und Ressourcen, die eigentständig in der Anwenung gestartet,
gestoppt, installiert und deinstalliert werden kann.
\index{Bundle}

Ressourcen und Klassen eines Bundles können anderen Bundles bereitgestellt
werden. Dazu müssen sie vom bereitstellenden Bundle explizit exportiert werden
und durch das nutzende Bundle explizit importiert werden. Jedes Bundle stellt einen
Bundlenamen, der üblicherweise von der enthaltenen Paketstruktur
abgeleitet wird, sowie eine Versionsnummer bereit. Aus diesen beiden Komponenten
wird dann eine eindeutige Identifizierung des Bundles generiert.
So ist es beispielsweise möglich, ein und das selbe Bundle in unterschiedlichen
Versionierungen zu installieren.

Über die \first{MANIFEST.MF}, die ohnehin in jedem Jar-File vorhanden ist, wird
das Bundle neben dem Namen und der Version mit weiteren Meta-Informationen, wie
importierte und exportierte Pakete oder Lauzeitumgebung, ausgestattet.
\index{MANIFEST.MF}

Jedes Bundle ist innerhalb des Frameworks mit einem eigenen Class-Loader
ausgestattet, über welchen ausschliesslich die Bundle-eigenen Klassen geladen
werden. Auf diese Weise sind die einzelen Bundles strikt voneinander getrennt und die
Import-Export-Beziehungen zwischen den Bundles können explizit gesteuert
werden. 
Neben anderen Vorteilen, wie etwa das Installieren des selben Bundles in
unterschiedlichen Versionen, wird auf diese Weise das vollständig
dynamische Hinzufügen und Entfernen von Bundles erst ermöglicht, da der
Class Path des Class Loaders nach dessen Instanziierung nicht mehr verändert
werden kann.
\citep{wtherich_die_2008}
% selbe version 17
% importieren, exportieren 13,79
% class loading 89
% MANIFEST.MF 21
% lebenszyklen 23,55
% unterschied Bundle / Plug-In 41
\subsubsection{Services} % 93
Das OSGi Framework stellt gemäß der serviceorientierten Architektur eine
zentrale, bundleübergreifende \first{Service Registry} bereit, an der
sogenannte \first{Services} angemeldet und abgefragt werden können.
\index{Service Registry}
Ein Service ist in diesem Fall ein simples Java-Objekt, typischerweise ein
Interface, das unter dem Interface-Namen und einer optionalen Beschreibung an
der Registry angemeldet wird.
Der Zugriff auf die Registry kann durch jedes beliebige, im
Framework geladene Bundle erfolgen. Der Zugriff kann hier aktiv oder passiv
erfolgen.
\index{Bundle}

Bei einem aktiven Zugriff, also der Registrierung eines Services durch ein
Bundle, wird technisch gesehen ein Service-Namen, sowie eine Beschreibung des
Services an der Service Registry hinterlegt. Der Service-Name ist üblicherweise
der voll qualifizierende Klassenname des Service Interfaces, die Beschreibung
des Service ist eine Sammlung von Strings, die eine entsprechende Beschreibung
der bereitgestellten Funktion beinhalten.
\index{Service Interface}

Bei einem passiven Zugriff durch ein beliebiges Bundle, also dem Abfragen eines
Service über dessen Namen oder Beschreibung, liefert die Registry eine
Referenz auf das entsprechende Interface zurück.
Über diese Referenz auf das Service Interface kann nun die Funktionalität, die
dieses Interface bereitstellt, genutzt werden, ohne dass hierbei bekannt ist,
wie diese Funktionalität implementiert ist, oder welches Bundle diese
bereitstellt.

Durch die dynamische Modularisierung des OSGi Frameworks können Services zur
Laufzeit \enquote{kommen und gehen}. Es liegt in der Verantwortung des
nutzenden Bundles, auf die aktuelle Verfügbarkeit des Service entsprechend zu
reagieren.

Die \first{OSGi Service Platform} enthält aufbauend auf das OSGi Framework eine
Reihe von \first{Standard Services}, mit denen häufig wiederkehrenden
Problemstellungen begegnet werden kann. So steht beispielsweise ein Log-Service
zur Verfügung, über den Bundles Nachrichten absenden und empfangen können.
Desweiteren sind über die Standard Services auch Funktionenen zur
Benutzerverwaltung, HTTP oder Datenbankanbindung realisiert.
\index{OSGi Service Platform}
% Standard Services 13,25
\citep{wtherich_die_2008}

\begin{figure}[htbp]
	\begin{center}
		\includegraphics[scale=1.3]{pics/osgi_layer.png}
	\caption[OSGi Schichtenmodel]{
	\textbf{OSGi Schichtenmodel}
	}
	\end{center}
	\label{fig:osgi_layer}
\end{figure}

Die serviceorientierte Architektur der Pipeline wird also durch das
\name{OSGi Framework} diktiert und stellt somit die unterste Schicht der
Architekturabstraktion der Pipeline dar.

