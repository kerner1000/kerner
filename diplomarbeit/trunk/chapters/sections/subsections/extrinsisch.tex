\subsection{extrinsisch \todo{geklaut von BPS}}
Die extrinsische Genvorhersage versucht, Gene oder Genkomponenten aufgrund von
Sequenzähnlichkeiten zu bereits bekannten Genen zu identifizieren.
\index{extrinsisch}
\index{Genvorhersage}
Dazu können verschiedene Datenbanken angefragt werden, wie z.B. \name{GenBank}.
Dabei können Vergleiche auf unterschiedlichen Ebenen herangezogen werden:
Zum einen der direkte Vergleich der genomischen Sequenz, also der eigentlichen
Basenabfolge, zum anderen kann die Basenabfolge in das daraus potentiell
resultierende Proteinprodukt übersetzt werden, mit welchem dann der eigentliche
Vergleich angestellt wird.

Ein großer Vorteil der extrinsischen Genvorhersage besteht darin, dass, wenn ein
signifikant ähnliches Gen gefunden wurde, diesem oft schon eine bestimmte
Funktion zugeordnet ist, die dann gegebenenfalls übernommen werden kann.
\index{extrinsisch}
\index{Genvorhersage}
Die funktionelle Annotation, also die Bestimmung der biologischen Funktion des
Gens, wird so schon teilweise abgedeckt.
\index{Annotation!funktionell}
Des weiteren sind Sequenzvergleiche auf Basis der Aminosäuresequenz im
Gegensatz zu Vergleichen auf Basis der genomischen Sequenz oftmals robuster
aufgrund von fehlendem Rauschen durch funktionsneutrale Nucleotid-Mutationen.
\index{Sequenzvergleich}

Der offensichtliche Nachteil der extrinsischen Genvorhersage besteht darin, dass
sie extensiell von den verfügbaren Datensätzen abhängt.
Ist bisher kein Gen bekannt, das Ähnlichkeit zu einem auf der zu annotierenden
Sequenz vorhandenen Gen aufweist, kann dessen Existenz durch die extrinsische
Annotation nicht aufgedeckt werden.

Die Sensitivität der extrinsischen sowie der funktionellen Annotation im
speziellen, kann deutlich verbessert werden, wenn eine musterbasierte, auch
patternbasierte Suche eingeschlossen wird.
\index{Suche!patternbasiert}
\index{Sensitivität}
\index{Annotation!funktionell}
\index{extrinsisch}
\index{musterbasiert}
\index{patternbasiert|see{musterbasiert}}
Die Proteinfunktion definiert sich maßgeblich über die auf dem Protein
enthaltenen Domänen.
\index{Domäne|see{Proteindomäne}}
\index{Proteindomäne}
Werden z.B. diese Domänen in Pattern übersetzt und in einer entsprechenden
Datenbank bereitgestellt, kann eine vergleichende Suche gegen diese Pattern
durchgeführt werden.
\index{Pattern}
\index{Domäne}
Eine ebensolche Datenbank ist \name{InterPro}, die über das Internet
erreichbar ist.
\index{Datenbank}
\index{InterPro}
Hier können Sequenzen auf in der Datenbank enthaltene Pattern
durchsucht werden und so potentiell enthaltene Domänen aufgespürt werden.
Diese patternbasierte Suche ist allerdings sehr zeitaufwändig und daher für
längere Sequenzen > 1kbp ineffzient.
\index{Suche!patternbasiert}
Für die Verifizierung und Spezifizierung bereits aufgeklärter Genstrukturen ist
die patternbasierte Suche trotzdem sehr gut geeignet.
\citep{pmid16749184}