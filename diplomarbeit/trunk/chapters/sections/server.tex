\section{Server}
Der Server ist über zwei OSGi Services \seec{chp:osgi}
realisiert:
Der \first{StepExecutor Service} sowie der \first{DataProxy Service}.
\index{Pipelineserver}
\index{OSGi Service}
\index{StepExecutor Service}
\index{DataProxy Service}
Der Zugriff auf den Server erfolgt somit über die \name{OSGi Service
Registry}.

\subsection{DataProxy Service}
Der \first{DataProxy Service} stellt die Referenz auf den \name{DataProxy}
bereit.
Dieser Service muss allerdings nicht duch die \name{Steps} selbst abgefragt
werden. Die Call-Back-Methoden von \code{Step} erhalten als Parameter bereits
diese Referenz.

\subsection{StepExecutor Service}
Der \name{StepExecutor Service} implementiert Methoden zur An- und Abmeldung
der \name{Steps} am Server.

\lstinputlisting[frame=single,label=stepExecutorService,caption=Das
Interface \code{StepExecutorService}]{code/server}

Ein \name{Step} wird typischerweise nach seiner Aktivierung
über die \name{Service Registry} den \name{StepExecutor Service} abfragen, und
sich selbst über dessen \code{registerStep} Methode am Server anmelden.
Nach der Anmeldung des \name{Steps} erstellt der Server einen neuen Thread, der
den \name{Step} gegebenenfalls zur Ausführung bringt.
Dabei wird zunächst überprüft, ob der Step zu diesem Zeitpunkt überhaupt
ausgeführt werden muss.
Die Methode \code{canBeSkipped}, die durch Step implementiert wird, gibt
darüber Auskunft. 
Typischerweise wird diese Methode \code{false} zurückgeben, wenn das zu
erwartende Ergebnis bereits in \name{DataBean} vorhanden ist.
Das ist beispielsweise der Fall, wenn dieser Step bei einem vorherigen Lauf,
der allerdings nicht vollständig beendet wurde, die Ergebnisse schon in
\name{DataBean} abgelegt hat.
\index{DataBean}
\index{Methode!boolean canBeSkipped(DataProxy d)}

Liefert \code{canBeSkipped} \code{false}, so werden als nächstes die nötigen
\name{preconditions} \seec{chp:softwarearchitektur} geprüft.
Dies geschieht über die Methode \code{requirementsSatified}.
Die Stepimplementierung sollte an dieser Stelle überprüfen, ob alle für die
Ausführung benötigten Daten bereits in \name{Databean} abgelegt sind.
\index{Methode!boolean requirementsSatisfied(DataProxy d)}

Liefert \code{requirementsSatisfied} \code{false}, wird der ausführende Thread
schlafen gelegt, bis ein anderer Step erfolgreich beendet wurde.
Daraufhin wird eine erneute Prüfung der \name{preconditions} durchgeführt.

Dieses Szenario wird solange wiederholt, bis \code{requirementsSatisfied}
\code{true} liefert.
Die eigentliche Auführung des Steps kann nun beginnen.
\index{Methode!boolean requirementsSatisfied(DataProxy d)}
Der Server ruft dazu die Methode \code{run} auf, die nach einer erfolgreichen
Ausführung \code{true} zurückliefert und die gewonnenen Ergebnisse in
\name{DataBean} ablegt.
\index{Methode!boolean run(DataProxy d)}

