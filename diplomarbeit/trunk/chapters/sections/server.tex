\section{Server}
Der Pipelineserver \todo{naming} ist über zwei OSGi Services realisiert:
Der \first{StepExecutor Service} sowie der \first{DataProxy Service}.
\index{Pipelineserver}
\index{OSGi Service}
\index{StepExecutor Service}
\index{DataProxy Service}
StepBundles \todo{naming} erlangen somit über die OSGi Service Registry
Zugriff auf die Funktionen des Servers.

\subsection{DataProxy Service}
Der \first{DataProxy Service} steuert den Zugriff auf den zentralen
\name{DataProxy}.
\index{DataProxy}
\index{DataProxy Service}
Die Methoden von \name{Step} bekommen als Parameter eine Referenz auf den
\name{DataProxy}.
Implementierungen von \code{Step} müssen daher nicht den \name{DataProxy
Service} von der Service registry abfragen.

\subsection{StepExecutor Service}
Der StepExecutor Service implementiert Methoden zur An- und Abmeldung eines
Steps am Pipelineserver \todo{naming}.
\index{Pipelineserver}
\index{StepExecutor Service}
\lstinputlisting[frame=single,label=server,caption=Server]{code/server}
Ein Step Bundle \todo{naming} wird typischerweise nach seiner Aktivierung über
die Service Registry den StepExecutor Service abfragen, und sich selbst über
dessen \code{registerStep} Methode am Server anmelden.
Nach der Anmeldung des Steps erstellt der Server einen neuen Thread, der den
Step gegebenenfalls zur Ausführung bringt.
Dabei wird zunächst überprüft, ob der Step zu diesem Zeitpunkt überhaupt
ausgeführt werden muss.
Die Methode \code{canBeSkipped}, die durch Step implementiert wird, gibt
darüber Auskunft. 
\lstinputlisting[frame=single,label=step,caption=Step]{code/step}
Typischerweise wird diese Methode \code{false} zurückgeben, wenn das zu
erwartende Step result \todo{naming} bereits im \name{DataBean} vorhanden ist.
Das ist beispielsweise der Fall, wenn dieser Step bei einem vorherigen Lauf,
der allerdings nicht vollständig beendet wurde, die Ergebnisse schon im
\name{DataBean} abgelegt hat.
\index{DataBean}
\index{Methode!boolean canBeSkipped(DataProxy d)}

Liefert \code{canBeSkipped} \code{false}, so werden als nächstes die nötigen
\first{execution preconditions} geprüft.
\index{execution precondition}
Dies geschieht über die Methode \code{requirementsSatified}.
Die Stepimplementierung sollte an dieser Stelle überprüfen, ob alle für die
Ausführung benötigten Daten bereits im \name{Databean} abgelegt sind.
\todo{Umsetzung von DataBean, DataProxy}
\index{Methode!boolean requirementsSatisfied(DataProxy d)}
\index{DataBean}

Liefert \code{requirementsSatisfied} \code{false}, wird der ausführende Thread
schlafen gelegt, bis ein anderer Step erfolgreich beendet wurde.
Daraufhin wird eine erneute Prüfung der requirements durchgeführt.
Dieses Szenario wird solange wiederholt, bis \code{requirements Satisfied}
\code{true} liefert.
Die eigentliche Auführung des Steps kann nun beginnen.
\index{Methode!boolean requirementsSatisfied(DataProxy d)}
Der Server ruft dazu die Methode \code{run} auf, die nach einer erfolgreichen
Ausführung \code{true} zurückliefert und die gewonnenen Ergebnisse im
\name{DataBean} ablegt.
\index{Methode!boolean run(DataProxy d)}
\index{DataBean}