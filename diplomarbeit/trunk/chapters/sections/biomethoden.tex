\section{bioinformatische Methoden}
Wie bereits erläutert \seec{chp:softwarearchitektur}, sollen die in
der Anwendung verwendeten Methoden zur Annotation dynamisch als \name{Step}
implementiert werden. Die Pipeline ist somit nicht auf bestimmte Methoden
beschränkt, vielmehr sind diese direkt durch die in der Pipeline verwendeten
Steps definiert \seef{fig:pipesFilter21}.

Aufgrund der konkreten Problemstellung durch die Zielsetzungen bezüglich der
Annotation von \name{C. higginsianum}, sowie als \name{Proof of Concept},
sollen einige, speziell auf die Annotation von \name{C. higginsianum} und den
hierzu verfügbaren Daten abgestimmte Steps implementiert werden.
\index{\name{C. higginsianum}|see\name{Colletotrichum higginsianum}}
Dabei soll die Annotation vorrangig \first{intrinsisch} \seec{chp:intrinsisch}
erfolgen, da homologiebasierte Ansätze in diesem Fall aus verschiedenen Gründen
wenig vielversprechend sind:
\index{intrinsisch}
\index{homologiebasiert}
\begin{itemize}
\item \todo{datenbanken haben nix}
\item \todo{ähm\ldots wars das schon?}
\end{itemize}
Für die intrinsische Genvorhersage sollen \first{Conditional Random
Fields}, oder \first{Hidden Markov Models} zum
Einsatz kommen.
\index{Conditional Random Field}
\index{Hidden Markov Model}
Diese Verfahren, insbesondere \name{Conditional Random Fields}, stellen zum
jetztigen Zeitpunkt die beste Möglichkeit zur intrinischen Annotation dar.
\todo{mehr?}

Das eigentliche \name{gene finding} soll durch verschiedene weitere Ansätze
unterstützt, bzw, verfeinert werden:

\begin{itemize}
  \item \todo{mask repetitive elemente}
  \item \todo{EST mapping}
  \item \todo{andere, maybe BLAST?}
\end{itemize}



