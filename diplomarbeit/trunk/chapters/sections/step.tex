\section{Step und AbstractStep}
Ein \name{Step} implementiert das gleichnaminge Interface, das drei
Call-Back-Methoden bereitstellt, die nach der Anmeldung am Server durch diesen
ausgeführt werden.
\index{Interface!Step}
Als Parameter erhalten diese Methoden jeweils eine Instanz auf den
pipelinezentralen \first{DataProxy} \seec{chp:data}.
\index{DataProxy}
\index{DataBean}

\lstinputlisting[frame=single,label=step,caption=Das Step Interface]{code/step}

\code{AbstractStep} stellt einen abstrakten Prototyp für eine Implementierung
von \code{Step} bereit.
\index{Klasse!AbstractStep}
Diese Klasse enthält bereits wiederkehrende Funktionalitäten wie das
Anmelden am Server, das Bereitstellen eines Loggers oder die Verwaltung von
Properties.
\code{AbstractStep} ist so konzipiert, dass alle Aspkete, die spezifisch für
das OSGi Framework sind, wie das Beziehen der Referenzen für den
Server oder den \name{DataProxy}, bereits in dieser Klasse
implementiert sind.
Wird eine neuer \name{Step} entworfen, der von \code{AbstractStep} erbt, müssen
nur die drei Methoden des Interfaces \code{Step} implementiert werden.
Ein neu entworfener \name{Step} kann sich somit vollständig auf
die Implementierung der eigentlichen Funktionaliät innerhalb der Pipeline konzentrieren.
Es entsteht kein programmatischer Overhead durch das OSGi
Framework.


